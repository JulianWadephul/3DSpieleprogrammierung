%************************************************
\chapter{Einleitung}\label{ch:einleitung}
%************************************************
Im Rahmen des Seminars "Programmieren 3: Betriebliche Informationssysteme" soll in dieser Seminararbeit die Thematik der 3D-Spieleentwicklung vorgestellt werden. Das Projekt soll jedoch nicht nur in einer schriftlichen Ausarbeitung, sondern auch in einem Workshop mit den Seminarteilnehmern behandelt werden. In dem Workshop gilt es die grundsätzlichen theoretischen Hintergrundinformationen darzustellen, und in einem praktischen Teil anhand von Aufgaben selbst etwas zu programmieren. Um den Seminarteilnehmern einige Inhalte zu vermitteln, haben wir uns entschlossen ein eigenes 3D-Spiel zu programmieren, welches im Workshop vervollständigt werden soll. Neben der veranschaulichten Darstellung der zu lernenden Inhalte, zeigt unser eigenes 3D-Spiel auf, was mit Java in der 3D-Spieleprogrammierung möglich ist. In dieser Seminararbeit werden wir die Grundlagen der 3D-Spieleprogrammierung behandeln, und dies anhand unseres 3D-Spiels veranschaulichen. Im folgenden möchten wir unser Spiel kurz vorstellen.\\

Unser entwickeltes Spiel 'Progman' ist eine Anlehnung an das bekannte Horror-Spiel 'Slenderman'. In unserer modifizierten Version geht es darum, dass der Spielende sich in einem Wald befindet und dort die 9 Bücher über anderen Themen in dem Seminar finden muss. Dabei wird er jedoch von einer Figur verfolgt, dem Progman, welcher versucht den Spielenden zu fangen. Ganz wesentlich hierbei ist die gruselige Stimmung, die durch Licht, Sound und Modelle in der Welt generiert wird. Im Laufe des Spiels nähert sich der Progman immer mehr an, bis er den Spielenden gefunden hat. Dabei gehen Faktoren wie die Anzahl der bereits eingesammelten Bücher und die häufige Benutzung der Taschenlampe in die Geschwindigkeit des Progman mit ein. Gewonnen hat der Spieler, wenn er alle 9 Bücher eingesammelt hat, bevor er vom Progman gefangen wurde. \\

Für die Programmierung eines 3D-Spiels in Java stehen verschiedene Engines zur Verfügung. Diese Engines beinhalten vorgefertigte Klassen und Methoden, welche das Programmieren eines 3D-Spiels deutlich vereinfachen. Wir haben uns für die jMonkeyEngine entschieden, doch darauf möchten wir später noch genauer eingehen. Außerdem werden wir verschiedene Themen der Umsetzung im Programmcode untersuchen. Am Ende möchten wir noch die Optimierung eines 3D-Spiels beschreiben, da ein nicht-optimiertes 3D-Spiel sehr schnell zu aufwendig werden kann.



