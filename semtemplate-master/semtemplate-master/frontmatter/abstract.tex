%*******************************************************
% Abstract
%*******************************************************
%\renewcommand{\abstractname}{Abstract}
%\pdfbookmark[1]{Abstract}{Abstract}
\begingroup
\let\clearpage\relax
\let\cleardoublepage\relax
\let\cleardoublepage\relax


% English abstract is only printed when thesis language is English
\ifx\myLanguage\english

\chapter*{Abstract}
Short summary of the contents in English\dots a great guide by 
Kent Beck how to write good abstracts can be found here:  
\begin{center}
\url{https://plg.uwaterloo.ca/~migod/research/beckOOPSLA.html}
\end{center}

\vfill

\fi

\begin{otherlanguage}{ngerman}
\pdfbookmark[1]{Zusammenfassung}{Zusammenfassung}
\chapter*{Zusammenfassung}
In dieser Seminararbeit wird die 3D-Programmierung, genauer die 3D-Spieleprogrammierung in Java behandelt. Zur Vereinfachung des Prozesses wird eine Game - Engine namens jMonkeyEngine3 verwendet. Die inkludierte Entwicklungsumgebung stellt die Grundfunktionen von Spielen bereit und ermöglicht, dass sich der Entwickler auf das Spiel selbst konzentrieren kann.
Auf dieser Basis wurde ein kleines Spiel programmiert, anhand welchem die fundamentalen Ideen und Umsetzungen von 3D-Programmierung geschildert werden. Dabei wird detailiert auf spezielle Anwendungen und wichtige Grundvorgehensweisen eingegangen, welche in allen heutigen 3D Spielen notwendig sind.

	
\end{otherlanguage}

\endgroup			

\vfill